\chapter{Risks}

% Sprint Questions
In this section of the specification, a list of the most likely and most serious risks for the project will be provided, along with the probability of them becoming a problem and any contingency plans in place. The probability of a risk becoming a problem is rated low, medium or high, with low being the least likely and high being the most likely. Risk management involves identifying, developing contingency plans in case they do become problems, and monitoring the project to give early warnings of potential risks.

% the selected \gls{widget} are not sufficent to cover usecases -> high rist because we did focus on only one usecase
One potential risk for this project is the possibility of there being the wrong \glspl{widget}, due to the focus on one \gls{usecase} and the \glspl{widget} not being created in consultation with the relevant decision makers. This risk is medium as it has the potential to significantly impact the overall success of the project. If the wrong \glspl{widget} are developed, it could lead to delays and additional costs. To mitigate this issue the Sales Employee has been consulted to ensure that the \glspl{widget} are relevant to the \glspl{usecase}. As a contingency the Decision Makers have to be consulted to ensure that the \glspl{widget} are relevant to the \glspl{usecase}.

% the feasibilty was not checked -> medium risk as all features where designed with the capabilities of the caruso database in mind but that is just theoretical
A potential risk for this project is that the feasibility of the \glspl{widget} was not checked in practice. This risk is medium as it could have some impact on the project, but is not necessarily a showstopper. To reduce the impact of this issue, all \glspl{widget} were designed with the capabilities of the \gls{dataplace} in mind, specifically their data catalogue. In case of any issues, the \glspl{widget} can be adjusted to fit the capabilities of the \gls{dataplace} or be discared.

% the configuration could not be easy to understand -> low risk as the configuration was designed in cooperation with the configurator to be easy to understand as this is the main goal of the project
There is a low risk that the configuration may not be easy to understand. To mitigate this issue, the configuration has been designed in cooperation with the configurator. This ensures that the configuration is clear and easy to follow, reducing the probability of it becoming a problem. If the configuration does happen to be confusing, a contingency plan has been put in place to provide additional support and assistance to those who may be struggling to understand it. This includes providing additional resources and guidance to help individuals navigate the configuration more effectively or even providing additional training to help them understand the configuration.

% caruso is not intending to invest their own development time in this project -> low risk critical decision maker point of views are not considered the product would not be usefull for them
Caruso may not invest their own development time in this project. While this could potentially be a problem, there is no hard requirement for a specific set of functionality, so the risk of this becoming a real issue is relatively low. However, it is important to note that this risk could only become a problem after the timeline of this project has been completed. As such, a contingency plan is not currently within the scope of this project. If Caruso does not invest their own development time in the project, it may be necessary to reassess the situation and consider potential alternative solutions or approaches.

% (maybe) the customer needs to be trained -> low risk as the data is designed to be easy to undestand as this is the main goal of the project

% (maybe) the configurator needs to be trained -> low risk as the configurator is designed in cooperation with the configurator 