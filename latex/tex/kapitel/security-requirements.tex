\chapter{Security Requirements}

This chapter covers the access and privacy requirements for the product. Section \ref{sec:access} specifies how cars and decision makers are connected as well as what cars the configurators can see. Section \ref{sec:privacy} goes over the consent management process and how it is handled for this project.

\section{Access Requirements}
\label{sec:access}
% decision makers should have access to their cars connected via the Caruso dataplace and should be able to delegate access to other employees of their company. For that the Caruso dataplace has a company organization unit that includes cars and employees.
Decision makers should have access to their cars that are connected via the \gls{dataplace} and should also be able to delegate access to other employees within their company. To facilitate this, the \gls{dataplace} has a company organization unit that includes both cars and employees. This unit allows decision makers to manage access to their connected cars and assign appropriate permissions to other employees within their organization. This helps to ensure that the necessary individuals have access to the data and functionality they need, while also maintaining control over who is able to access sensitive information and perform certain actions.

% Configurators dont have access to the cars registered to a decision maker. They can only access the cars they are assigned to. This has to be kept in mind when designing the configurator, it should include some example cars that can be used to test the configurator. One can use the VirtualOEM from caruso.
It is important to keep in mind that configurators do not have access to the cars registered to a decision maker. They can only access the cars that they are specifically assigned to. This should be considered when designing the configurator's view, as it may be necessary to include some example cars that can be used for testing purposes. One option for this is to use the VirtualOEM from Caruso, which provides a set of sample cars that can be used for testing and development. This will help to ensure that configurators are able to effectively test and debug their configurations without needing access to actual cars.

\section{Privacy Requirements}
\label{sec:privacy}
% consent management is handled by caruso and it not subject of this project
One important aspect of the privacy requirements in an application concerning connected cars is consent management. Consent management refers to the processes and systems in place for obtaining and documenting an individual's consent for the collection, use, and sharing of their personal data. In the case of Carvis, consent management is handled by Caruso and is therefore not subject of this project.
