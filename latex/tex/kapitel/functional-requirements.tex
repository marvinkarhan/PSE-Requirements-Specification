\chapter{Functional Requirements}

  \sffamily
  \begin{footnotesize}
    \begin{longtable}[i i L]{ p{.1\textwidth} p{.1\textwidth} p{.5\textwidth} }
      \caption                       % Caption für das Tabellenverzeichnis
          {Entscheidungsträger} % Caption für die Tabelle selbs
          \\
      \toprule
      \textbf{US ID} & \textbf{PUC} & \textbf{Description} \\
      \midrule
      US\_01 & & As a Decision Maker, I want to see a list of all of my vehicles to see which vehicles are in my fleet.\\
      US\_02 & & As a Decision Maker, I want to add a vehicle to my fleet to account to changes to my cars. \\
      US\_03 & & Als Entscheidungsträger will ich sehen, ob mein Fahrzeug unterwegs ist oder steht, um zu wissen, ob das Fahrzeug derzeit verwendet wird. \\
      US\_04 & & Als Entscheidungsträger will ich die VIN meines Fahrzeugs in der Liste meiner Fahrzeuge sehen, um mein Fahrzeug eindeutig zu identifizieren. \\
      US\_05 & & Als Entscheidungsträger will ich Fahrzeuge anhand der, in der Fahrzeugliste vorhandenen Kriterien filtern können, um nur Informationen angezeigt zu bekommen, die mich aktuell interessieren \\
      US\_06 & & Als Entscheidungsträger will ich einen Bericht zu einem aktuell ausgewählten Fahrzeug herunterladen, damit ich diesen persistieren kann. \\
      US\_07 & & Als Entscheidungsträger will ich einen Bericht über alle meine Fahrzeuge herunterladen, um diesen mit anderen Mitarbeitern zu teilen. \\
      \bottomrule
    \end{longtable}
  \end{footnotesize}
  \rmfamily


  \sffamily
  \begin{footnotesize}
    \renewcommand{\arraystretch}{1.4}
    \begin{longtable}[i i L]{ p{.1\textwidth} p{.1\textwidth} p{.7\textwidth} }
      \caption                       % Caption für das Tabellenverzeichnis
          {Konfigurator} % Caption für die Tabelle selbs
          \\
      \toprule
      \textbf{US ID} & \textbf{PUC} & \textbf{Description} \\
      \midrule

      USC01 & & Als Konfigurator will ich ein neues Projekt für einen Entscheidungsträger erstellen, damit ich für diesen eine individuelle Präsentation vorbereiten kann. \\
      USC02 & & Als Konfigurator will ich ein Marketplace-Konto mit einem Projekt verknüpfen, um die Fahrzeuge des Kontos mit der Anwendung zu verbinden. \\
      USC03 & & Als Konfigurator will ich mehrere Marketplace-Konten zum Betrachten eines Projekts hinzufügen, um mehreren Mitarbeitern des Entscheidungsträgers Zugriff auf den Data Appettizer zu gewähren. \\
      USC04 & & Als Konfigurator will ich das Logo für einen Entscheidungsträger in seinem Projekt hinterlegen, damit dieser mehr Vertrauen in die Anwendung erhält. 
      \textbf{Anmerkung:} Das Logo soll sichtbar bei Verwendung des Appetizers sein. \\
      USC05 & & Als Konfigurator will ich eine Hauptfarbe und eine Akzentfarbe für das Projekt eines Entscheidungsträgers einstellen, damit dessen CI seines Unternehmens im Appetizer vertreten wird. \\
      USC06 & & Als Konfigurator will ich einsehen, wer und wann die letzte Änderung an einem Projekt durchgeführt wurde, um nachvollziehen zu können, ob was geändert wurde. \\
      USC07 & & Als Konfigurator will sehen, wer und wann das letzte Mal das Projekt eingesehen wurde, um zu wissen, ob das Projekt noch verwendet wird. \\
      USC08 & & Als Konfigurator will ich ein Projekt als Vorlage speichern, um dieses für ein weiteres Projekt als Vorlage zu verwenden \\
      USC09 & & Als Konfigurator will ich eine Vorkonfiguration in Form einer Vorlage für ein Projekt wählen können, um nicht alle Einstellungen manuell durchführen zu müssen \\
      USC10 & & Als Konfigurator will ich eine Liste aller meiner Vorlagen ansehen, um diese für ein Projekt auswählen zu können. \\
      USC11 & & Als Konfigurator will ich eine Vorlage anhand des Namens oder den verwendeten Datenpunkten suchen können, um schnell eine passende Vorlage für mein Projekt zu finden. \\
      USC12 & & Als Konfigurator will ich nach Projekten anhand ihres Namens, der Anzahl von Autos, oder dem Besitzer suchen, um schnell das richtige Projekt zu finden \\
      USC13 & & Als Konfigurator will ich einstellen, welche Informationen in der Liste der Fahrzeuge zu sehen sind, um den Wünschen des Entscheidungsträgers für ihn relevante Informationen auf einen Blick zur Verfügung zu stellen.

      \textbf{Anmerkung:} Zur Auswahl stehen müssen VIN, Hersteller, Status, Kilometerstand, das nächste Ereignis, Modell und Tankstand \\
      USC14 & & Als Konfigurator will ich eine Statistik über die Anzahl der in der Flotte vorhandenen Hersteller für den Entscheidungsträger hinzufügen können, damit dieser einen Überblick über alle Fahrzeughersteller in seiner Flotte bekommt \\
      USC15 & & Als Konfigurator will ich eine chronologische Liste aller Fahrzeugereignisse über alle Autos für den Entscheidungsträger hinzufügen, damit dieser schnell erkennt, welches Fahrzeug Probleme hat 
      
      \textbf{Anmerkung:} Fahrzeugereignisse sind Unfälle oder aktive Kontrollleuchten \\ 
      USC16 & & Als Konfigurator will ich eine Statistik über die Motortypen der Flottenfahrzeuge für den Entscheidungsträger hinzufügen, damit dieser besser planen kann, welche Fahrzeuge wie betankt werden müssen. 
      
      \textbf{Anmerkung:} Motortypen sind Benzin, Elektro oder Hybrid.
      \\
      USC17 & & Als Konfigurator will ich eine Statistik über alle gefahrenen Kilometer aller Flottenfahrzeuge für den Entscheidungsträger hinzufügen, damit dieser weiß, welche Strecken an welchen Tagen hinterlegt wurden. \\
      USC18 & & Als Konfigurator will ich eine Karte mit allen Fahrzeugen für den Entscheidungsträger hinzufügen, damit dieser die Position aller seiner Fahrzeug einsehen kann. \\
      USC19 & & Als Konfigurator will ich meinen letzten Hinzufügen Schritt rückgängig machen können, um Fehler schneller zu beheben. \\
      USC20 & & Als Konfigurator will ich meinen letzten rückgängig Schritt wiederholen könne, um Fehler zu beheben. \\
      USC21 & & Als Konfigurator will ich hinzugefügte Widgets entfernen können, um die Ansicht für den Entscheidungsträger zu konfigurieren. \\

      \bottomrule
    \end{longtable}
  \end{footnotesize}
  \rmfamily
% blub