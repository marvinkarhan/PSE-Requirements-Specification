\chapter{Functional Requirements}
Zur Erhebung der Anforderungen wurden User Stories entworfen, die den den jeweiligen Stakeholder, seine Wünsche und seine Ziele beinhalten. Die folgenden Tabellen beinhalten die Anforderungen gruppiert nach den einzelnen Stakeholdern der Core Target Group. Jede User Story besitzt eine ID, sowie eine Referenz auf die dazugehörigen Use Cases auf, welche sie sich beziehen. Außerdem beinhalten sie eine Beschreibung, welche die User Story im Detail darstellt. Die Priorisierung der User Story ist anhand der MoSCoW-Priorisierung erfolgt. Die MoSCoW-Priorisierung beinhaltet Muss, Soll und Kann Kriterien. 
% TODO @JAD elaborieren pls %

  \sffamily
  \begin{footnotesize}
    \begin{longtable}[L L L L]{ p{.1\textwidth} p{.1\textwidth} p{.6\textwidth} p{.1\textwidth}}
      \caption                       % Caption für das Tabellenverzeichnis
          {User Stories des Entscheidungsträgers und Sales Employees} % Caption für die Tabelle selbs
          \\
      \toprule
      \textbf{US ID} & \textbf{UC} & \textbf{Description} & \textbf{Criteria} \\
      \midrule

      \hypertarget{Ref:US1}{US\_01} & \hyperlink{Ref:UC1}{UC\_01} & As a Decision Maker and Sales Employee, I want to see a list of all of my vehicles to see which vehicles are in my fleet. & Must \\ 

      \hypertarget{Ref:US2}{US\_02} & \hyperlink{Ref:UC1}{UC\_01} & Als Entscheidungsträger und Sales Employee will ich die Anzahl der Fahrzeuge sehen, die derzeit mit meinem Marketplace-Konto verbunden sind, um zu erkennen wie viel Fahrzeuge ich habe. & Should \\

      \hypertarget{Ref:US3}{US\_03} & \hyperlink{Ref:UC1}{UC\_01} & Als Entscheidungsträger und Sales Employee will ich ein Fahrzeug suchen, um es in der Liste der Fahrzeuge zu finden.
      \newline\newline
      \emph{Die Suchkriterien entsprechen den Informationen, die in der Fahrzeugtablle vorhanden sind.} & Should
      \\

      \hypertarget{Ref:US4}{US\_04} & \hyperlink{Ref:UC1}{UC\_01} & Als Entscheidungsträger und Sales Employee will ich Fahrzeuge in der Fahrzeugtabelle filtern, um nur die Fahrzeuge angezeigt zu bekommen, die meinen Kriterien entsprechen.
      \newline\newline
      \emph{Die Filterkriterien entsprechen den Spalten in der Fahrzeugtabelle.}
      & Should
      \\

      \hypertarget{Ref:US5}{US\_05} & \hyperlink{Ref:UC1}{UC\_01} & Als Entscheidungsträger und Sales Employee will ich Fahrzeuge in der Fahrzeugtabelle sortieren, um die Fahrzeuge gruppiert anzeigen zu können. 
      \newline\newline
      \emph{Die Sortierung soll ab und aufsteigend und für alle Spalten in der Fahrzeugtabelle möglich sein.}
      & Should
      \\

      \hypertarget{Ref:US6}{US\_06} & \hyperlink{Ref:UC1}{UC\_01}, \newline \hyperlink{Ref:UC2}{UC\_02}, \newline \hyperlink{Ref:UC3}{UC\_03} & Als Entscheidungsträger und Sales Employee will ich meine Liste von Fahrzeugen, meine fahrzeugspezifischen Informationen sowie die Statistik über die gesamte Flotte als PDFs herunterladen, um die Daten anderen Mitarbeitern zu zeigen. & Can \\

      \hypertarget{Ref:US7}{US\_07} & \hyperlink{Ref:UC2}{UC\_02} & Als Entscheidungsträger und Sale Employee will ich Statistiken und aktuelle Fahrzeuginformationen zu einem einzelnen Fahrzeug einsehen können, um ein Verständnis für die Daten von Caruso zu erhalten. 
      \newline\newline
      \emph{Die angezeigten Datenpunkte werden in Form Widgets vom Sales Mitarbeiter festgelegt. Eine Liste der möglichen Widgets ist in Tabelle \ref{FahrzeugWidgets} vorhanden. Dies verringert die Redundanz der User Stories, da sie nicht für den Entscheidungsträger wiederholt werden müssen.}
      & Must \\

      \hypertarget{Ref:US8}{US\_08} & \hyperlink{Ref:UC2}{UC\_02} & Als Entscheidungsträger und Sales Employee will ich von einem fahrzeugspezifischen Widget angezeigt bekommen, ob das aktuelle Fahrzeug den Datenpunkt unterstützt, damit ich weiß, ob Daten über dieses Fahrzeug zur Verfügung stehen.
      & Must
      \\

      \hypertarget{Ref:US9}{US\_09} & \hyperlink{Ref:UC2}{UC\_02}, \newline \hyperlink{Ref:UC3}{UC\_03} & Als Entscheidungsträger und Sales Employee will ich wissen, wann die Informationen des Widgets das letzte Mal aktualisiert wurden, um mich über die Aktualität der Daten zu vergewissern.
      & Must
      \\

      \hypertarget{Ref:US10}{US\_10} & \hyperlink{Ref:UC3}{UC\_03} & Als Entscheidungsträger und Sales Employee will ich Statistiken und aktuelle Fahrzeuginformationen zu meiner gesamten Flotte auf einen Blick einsehen, um ein Verständnis über die gesamte Flotte und deren Zustand zu erhalten.
      \newline\newline
      \emph{Die angezeigten Datenpunkte werden in Form Widgets vom Sales Mitarbeiter festgelegt. Eine Liste der möglichen Widgets ist in Tabelle \ref{FlottenWidgets} vorhanden. Dies verringert die Redundanz der User Stories, da sie nicht für den Entscheidungsträger wiederholt werden müssen.} & Must \\

      \hypertarget{Ref:US11}{US\_11} & \hyperlink{Ref:UC3}{UC\_03} & Als Entscheidungsträger und Sales Employee will ich sehen, wie viele Fahrzeuge von allen möglichen Fahrzeugen für ein flottenspezifisches Widget verwendet werden, damit ich weiß, aus welcher Menge die Daten erhoben werden. & Must \\

      \hypertarget{Ref:US12}{US\_12} & \hyperlink{Ref:UC3}{UC\_03} & Als Entscheidungsträger und Sales Employee will ich die Inhalte der flottenspezifischen Widgets filtern können, damit ich nur die Daten aus der Menge von Fahrzeugen erhalte, welche ich betrachten will.
      \newline\newline
      \emph{Die Filterkriterien entsprechen den Spalten aus der Fahrzeugtabelle.}
      & Should \\

      \bottomrule
    \end{longtable}
  \end{footnotesize}
  \rmfamily

  \sffamily
  \begin{footnotesize}
    \begin{longtable}[L L L L]{ p{.1\textwidth} p{.1\textwidth} p{.6\textwidth} p{.1\textwidth} }
      \caption                       % Caption für das Tabellenverzeichnis
          {User Stories des Sales Mitarbeiters} % Caption für die Tabelle selbs
          \\
      \toprule
      \textbf{US ID} & \textbf{UC} & \textbf{Description} & \textbf{Criteria} \\
      \midrule
      \hypertarget{Ref:US13}{US\_13} & \hyperlink{Ref:UC4}{UC\_04} & Als Sales Mitarbeiter will ich alle Projekte sehen, damit ich weiß, welche Projekte existieren. & Must \\
      \hypertarget{Ref:US14}{US\_14} & \hyperlink{Ref:UC4}{UC\_04} & Als Sales Mitarbeiter will ich Projekte anhand ihres Namens suchen können, um das passende Projekt zu finden. & Should \\
      \hypertarget{Ref:US15}{US\_15} & \hyperlink{Ref:UC4}{UC\_04} & Als Sales Mitarbeiter will ich die Anzahl meiner Projekte sehen, damit ich weiß, wie viele Projekte existieren. & Should \\
      \hypertarget{Ref:US16}{US\_16} & \hyperlink{Ref:UC4}{UC\_04} & Als Sales Mitarbeiter will ich sehen, wann und von wem die letzte Änderung an einem Projekt durchgeführt wurde, damit ich weiß, ob an dem Projekt gearbeitet wurde. & Must \\
      \hypertarget{Ref:US17}{US\_17} & \hyperlink{Ref:UC4}{UC\_04} & Als Sales Mitarbeiter will ich sehen, wann und wer das Projekt zuletzt eingesehen hat, um zu wissen, ob das Projekt noch verwendet wird & Must \\
      \hypertarget{Ref:US18}{US\_18} & \hyperlink{Ref:UC5}{UC\_05} & Als Sales Mitarbeiter will ich ein Projekt erstellen, um neue Präsentationen für Entscheidungsträger vorbereiten zu können. & Must \\
      \hypertarget{Ref:US19}{US\_19} & \hyperlink{Ref:UC6}{UC\_06} & Als Sales Mitarbeiter will ich ein Projekt bearbeiten, um Projekteinformationen nachträglich ändern zu können. & Must \\
      \hypertarget{Ref:US20}{US\_20} & \hyperlink{Ref:UC5}{UC\_05}, \newline \hyperlink{Ref:UC6}{UC\_06} & Als Sales Mitarbeiter will ich ein Marketplace-Konto mit einem Projekt verknüpfen, um die Fahrzeuge des Kontos mit der Anwendung zu verbinden. & Must \\
      \hypertarget{Ref:US21}{US\_21} & \hyperlink{Ref:UC5}{UC\_05}, \newline \hyperlink{Ref:UC6}{UC\_06} & Als Sales Mitarbeiter will ich mehrere Marketplace-Konten zum Betrachten eines Projekts hinzufügen, um mehreren Mitarbeitern des Entscheidungsträgers Zugriff auf das Projekt zu gewähren. & Should \\
      \hypertarget{Ref:US22}{US\_22} & \hyperlink{Ref:UC5}{UC\_05}, \newline \hyperlink{Ref:UC6}{UC\_06} & Als Sales Mitarbeiter will ich das Logo für einen Entscheidungsträger in seinem Projekt hinterlegen, damit dessen CI im Projekt wiedergefunden wird. 
      \newline\newline
      \emph{Das Logo soll sichtbar bei Verwendung der Applikation sein.} & Should
      \\
      \hypertarget{Ref:US23}{US\_23} & \hyperlink{Ref:UC5}{UC\_05}, \newline \hyperlink{Ref:UC6}{UC\_06} & Als Sales Mitarbeiter will ich eine Hauptfarbe und eine Akzentfarbe für das Projekt eines Entscheidungsträgers einstellen, damit dessen CI seines Unternehmens im Projekt vertreten wird. & Should \\
      \hypertarget{Ref:US24}{US\_24} & \hyperlink{Ref:UC5}{UC\_05}, \newline \hyperlink{Ref:UC8}{UC\_08} & Als Sales Mitarbeiter will ich eine Liste aller Vorlagen sehen, damit ich weiß welche Vorkonfigurationen für mein Projekt zur Verfügung stehen. & Must \\
      \hypertarget{Ref:US25}{US\_25} & \hyperlink{Ref:UC5}{UC\_05}, \newline \hyperlink{Ref:UC8}{UC\_08} & Als Sales Mitarbeiter will ich eine Vorschau für meine Vorlagen sehen, um zu vorher zu wissen, ob die Konfiguration für mein Projekt sinnvoll ist & Should \\
      \hypertarget{Ref:US26}{US\_26} & \hyperlink{Ref:UC5}{UC\_05}, \newline \hyperlink{Ref:UC8}{UC\_08} & Als Sales Mitarbeiter will ich eine Vorlage anhand des Namens suchen, um die Vorlage in der Liste der Vorlagen zu finden. & Should \\
      \hypertarget{Ref:US27}{US\_27} & \hyperlink{Ref:UC7}{UC\_07} & Als Sales Mitarbeiter will ich ein Projekt löschen, um nicht benötigte Projekte zu entfernen & Must \\
      \hypertarget{Ref:US28}{US\_28} & \hyperlink{Ref:UC8}{UC\_08} & Als Sales Mitarbeiter will ich ein Projekt zurücksetzen, um das Projekt neu zu konfigurieren. & Should \\
      \hypertarget{Ref:US29}{US\_29} & \hyperlink{Ref:UC9}{UC\_09} & Als Sales Mitarbeiter will ich ein Projekt als Vorlage speichern, um dieses für zukünftige Projekte zu verwenden. & Must \\
      \hypertarget{Ref:US30}{US\_30} & \hyperlink{Ref:UC10}{UC\_10} & Als Sales Mitarbeiter will ich neue Informationen zu der Fahrzeugtablle hinzufügen, um dem Entscheidungsträger für ihn relevante Informationen auf der Startseite anzuzeigen
      \newline\newline
      \emph{Die angezeigten Informationen sind die Spalten der Tabell. Zur Verfügung stehen müssen VIN, Hersteller, Status, Kilometerstand, das nächste Ereignis, Modell und Tankstand} & Must  \\
      \hypertarget{Ref:US31}{US\_31} & \hyperlink{Ref:UC10}{UC\_10} & Als Sales Mitarbeiter will ich Informationen von der Fahrzeugtablle entfernen, um für den Entscheidungsträger irrelevante Informationen zu versteckenn & Must  \\
      \hypertarget{Ref:US32}{US\_32} & \hyperlink{Ref:UC10}{UC\_10} & Als Sales Mitarbeiter will ich die Reihenfolge der Informationen verändern, um die Informationen für den Entscheidungsträger zu priorisieren & Should  \\

      \hypertarget{Ref:US33}{US\_33} & \hyperlink{Ref:UC11}{UC\_11} & Als Sales Mitarbeiter will ich eine Vorschau für ein Widget haben, dass ich hinzufügen will, um zu wissen, wie es aussieht bevor ich es hinzufüge. & Should \\
      \hypertarget{Ref:US34}{US\_34} & \hyperlink{Ref:UC12}{UC\_12} & Als Sales Mitarbeiter will ich ein Widget löschen, um nicht benötigte Informationen für den Entscheidungsträger zu entfernen. & Must \\
      \hypertarget{Ref:US35}{US\_35} & \hyperlink{Ref:UC13}{UC\_13} \  & Als Sales Mitarbeiter will ich ein Widget verschieben, damit ich die Reihenfolge für den Entscheidungsträger priorisieren kann. & Should \\
      \hypertarget{Ref:US36}{US\_36} & \hyperlink{Ref:UC11}{UC\_11}, \newline \hyperlink{Ref:UC12}{UC\_12}, \newline \hyperlink{Ref:UC13}{UC\_13} & Als Sales Mitarbeiter will meine letzte Aktion rückgängig machen, um Fehler zu korrigieren. & Should \\
      \hypertarget{Ref:US37}{US\_37} & \hyperlink{Ref:UC11}{UC\_11}, \newline \hyperlink{Ref:UC12}{UC\_12}, \newline \hyperlink{Ref:UC13}{UC\_13} & Als Sales Mitarbeiter will ich meine letzte Aktion wiederherstellen können, um versehentlich rückgängig gemachte Aktion zu korrigieren. & Should \\
      \bottomrule
    \end{longtable}
  \end{footnotesize}
  \rmfamily

  \sffamily
  \begin{footnotesize}
    \label{FahrzeugWidgets}
    \begin{longtable}[L L L]{ p{.1\textwidth} p{.1\textwidth} p{.6\textwidth} p{.1\textwidth} }
      \caption                       % Caption für das Tabellenverzeichnis
          {User Stories des Sales Mitarbeiters für fahrzeugspezifische Widgets} % Caption für die Tabelle selbs
          \\
      \toprule
      \textbf{US ID} & \textbf{UC} & \textbf{Description} & \textbf{Criteria} \\
      \midrule
      \hypertarget{Ref:US38}{US\_38} & \hyperlink{Ref:UC11}{UC\_11} & Als Sales Mitarbeiter will ich ein Widget für den Entscheidungsträger hinzufügen, dass die VIN des Fahrzeugs, den Hersteller und den aktuellen Kilometerstand anzeigt, damit der Entscheidungsträger weiß, welches Fahrzeug er derzeit betrachtet. & Must \\
      \hypertarget{Ref:US39}{US\_39} & \hyperlink{Ref:UC11}{UC\_11} & Als Sales Mitarbeiter will ich ein Widget für den Entscheidungsträger hinzufügen, dass den aktuellen Tankstand sowie die verbleibende Reichweite in Kilometer für das Fahrzeug anzeigt, damit der Entscheidungsträger weiß, ob das Fahrzeug betankt werden muss. & Must \\
      \hypertarget{Ref:US40}{US\_40} & \hyperlink{Ref:UC11}{UC\_11} & Als Sales Mitarbeiter will ich ein Widget für den Entscheidungsträger hinzufügen, dass die nächsten Termine für das Fahrzeug darstellt, damit der Entscheidungsträger weiß, ob das Fahrzeug demnächst in die Werkstatt muss. 
      \newline\newline
      \emph{Termine sind anstehende Servicetermine, Hauptuntersuchungen oder Ölwechsel.} & Must \\
      \hypertarget{Ref:US41}{US\_41} & \hyperlink{Ref:UC11}{UC\_11} & Als Sales Mitarbeiter will ich ein Widget für den Entscheidungsträger hinzufügen, dass die Position des Fahrzeugs, sowie seine aktuelle Geschwindigkeit auf einer Karte darstellt, damit der Entscheidungsträger weiß, wo sich das Fahrzeug befindet und wie schnell es sich derzeit bewegt. & Must \\
      \hypertarget{Ref:US42}{US\_42} & \hyperlink{Ref:UC11}{UC\_11} & Als Sales Mitarbeiter will ich ein Widget für den Entscheidungsträger hinzufügen, dass vergangen Fahrten sowie die Stopps des Fahrzeugs auf der Karte darstellt, damit der Entscheidungsträger weiß, wie seine Fahrzeuge in der Vergangenheit gefahren sind. & Must \\
      \hypertarget{Ref:US43}{US\_43} & \hyperlink{Ref:UC11}{UC\_11} & Als Sales Mitarbeiter will ich ein Widget für den Entscheidungsträger hinzufügen, dass eine Statistik der vergangenen gefahrenen Kilometer auf die Wochentage verteilt anzeigt, damit der Entscheidungsträger weiß, wie häufig das Fahrzeug verwendet wird. & Must \\
      \hypertarget{Ref:US44}{US\_44} & \hyperlink{Ref:UC11}{UC\_11} & Als Sales Mitarbeiter will ich ein Widget für den Entscheidungsträger hinzufügen, dass eine chronologische Liste der Fahrzeugeregnisse anzeigt, damit der Entscheidungsträger weiß, wann etwas mit dem Fahrzeug passiert ist. 
      \newline\newline
      \emph{Fahrzeugereignisse beziehen sich auf Unfälle oder Kontrollleuchten. Unfälle sollen hevorgehoben werden, damit der Entscheidungsträger direkt weiß, ob das Fahrzeug einen Schaden hat.} & Must
      \\
      \hypertarget{Ref:US45}{US\_45} & \hyperlink{Ref:UC11}{UC\_11} & Als Sales Mitarbeiter will ich ein Widget hinzufügen, dass eine Statistik über die vergangene Geschwindigkeit des Fahrzeugs anzeigt, damit der Entscheidungsträger weiß, wie sicher das Fahrzeug gefahren wird. & Must \\
      \hypertarget{Ref:US46}{US\_46} & \hyperlink{Ref:UC11}{UC\_11} & Als Sales Mitarbeiter will ich eine Widget hinzufügen, dass die aktuellen Kontrollleuchten des Fahrzeugs darstellt, damit der Entscheidungsträger weiß, ob aktuell etwas am Fahrzeug fehlerhaft ist. & Must \\
      \hypertarget{Ref:US47}{US\_47} & \hyperlink{Ref:UC11}{UC\_11} & Als Sales Mitarbeiter will ich ein Widget hinzufügen, dass aktuellen Fahrtinformationen anzeigt, damit der Entscheidungsträger weiß, wie lange das Fahrzeug unterwegs ist und mit welcher Geschwindigkeit es sich bewegt.
      \newline\newline
      \emph{Die aktuelle Fahrtinformation beinhaltet seit wann das Fahrzeug unterwegs ist und in welcher Straße es zuletzt gehalten hat.} & Must \\
      \hypertarget{Ref:US48}{US\_48} & \hyperlink{Ref:UC11}{UC\_11} & Als Konfigurator will ich ein Widget hinzufügen, dass den aktuellen Tankverbrauch des Fahrzeugs pro 100km anzeigt, damit der Entscheidungsträger weiß, wie hoch der Tankverbrauch seines Fahrzeugs ist. & Must \\

      \bottomrule
    \end{longtable}
  \end{footnotesize}
  \rmfamily

  \label{FlottenWidgets}
  \sffamily
  \begin{footnotesize}
    \begin{longtable}[L L L L ]{ p{.1\textwidth} p{.1\textwidth} p{.6\textwidth} p{.1\textwidth}}
      \caption                       % Caption für das Tabellenverzeichnis
          {User Stories des Konfigurators für flottenspezifische Widgets} % Caption für die Tabelle selbs
          \\
      \toprule
      \textbf{US ID} & \textbf{UC} & \textbf{Description} & \textbf{Criteria} \\
      \midrule
      \hypertarget{Ref:US49}{US\_49} & \hyperlink{Ref:UC11}{UC\_11} & Als Sales Mitarbeiter will ich ein Widget hinzufügen, dass eine Statistik über die Anzahl der in der Flotte vorhandenen Hersteller anzeigt, damit der Entscheidungsträger weiß, von welchem Hersteller sein Fahrzeuge sind. & Must  \\
      \hypertarget{Ref:US50}{US\_50} & \hyperlink{Ref:UC11}{UC\_11} & Als Sales Mitarbeiter will ich ein Widget hinzufügen, dass eine chronologische Liste aller Fahrzeugeregnisse über alle Fahrzeuge anzeigt, damit der Entscheidungsträger auf einen Blick erkennt, welches Auto vor kurzem ein Ereignis hatte.
      \newline\newline
      \emph{Fahrzeugereignisse beziehen sich auf Unfälle oder Kontrollleuchten. Unfälle sollen hevorgehoben werden, damit der Entscheidungsträger direkt weiß, ob das Fahrzeug einen Schaden hat.} & Must \\
      \hypertarget{Ref:US51}{US\_51} & \hyperlink{Ref:UC11}{UC\_11} & Als Konfigurator will ich ein Widget hinzufügen, welches eine Statistik über die Motortypen der Flottenfahrzeuge anzeigt, damit der Entscheidungsträger weiß, welche Motortypen seine Fahrzeuge beinhalten. 
      \newline\newline
      \emph{Motortypen sind Benziner, Elektromotoren oder Hybride.} & Must \\
      \hypertarget{Ref:US52}{US\_52} & \hyperlink{Ref:UC11}{UC\_11} & als Konfigurator will ich ein Widget hinzufügen, welches eine Statistik über alle gefahrenen Kilometer alle Flottenfahrzeuge anzeigt, damit der Entscheidungsträger weiß, wie welche Streckenlänge alle Fahrzeuge an welchen Tagen hinterlegen. & Must \\
      \hypertarget{Ref:US53}{US\_53} & \hyperlink{Ref:UC11}{UC\_11} & Als Konfigurator will ich ein Widget hinzufügen, welches eine Karte mit allen Fahrzeugen anzeigt, damit der Entscheidungsträger die Position aller seiner Fahrzeuge einsehen kann. & Must \\
      \hypertarget{Ref:US54}{US\_54} & \hyperlink{Ref:UC11}{UC\_11} & Als Konfigurator will ich ein Widget hinzufügen, welches den gesamten Kilometerstand aller Fahrzeuge zusammen und im Durchschnitt anzeigt, damit der Entscheidungsträger weiß, wie hoch der Gesamtkilometerstand seiner Fahrzeuge ist. & Must \\
      \bottomrule
    \end{longtable}
  \end{footnotesize}
  \rmfamily
% blub

      
