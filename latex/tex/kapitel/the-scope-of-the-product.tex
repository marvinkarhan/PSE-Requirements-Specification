\chapter{The Scope of the Product}
In diesem Kapitel werden die einzelnen Anwendungsfälle des Produkts vorgestellt. Dabei werden die Rollen des Entscheidungsträgers, Sales Mitarbeiters und Konfigurators betrachtet. Anschließend werden Beschreibungen für die einzelnen Uses Case aufgeführt und für jeden der Anwendungsfälle ein Walkthrough in Form eines High fidelity Prototypen dargestellt.
\section{Product Boundary}
Die folgende Abbildung \ref{SopeOfProduct:ContextDiagram} zeigt ein Anwendungsfalldiagramm für die Stakeholder der Core Target Group und ihren individuellen Anwendungsfällen mit dem Produkt Carvis. Alle Use Cases sind mit einer Id versehen, welche über die weiteren Kapitel referenziert werden.
\begin{figure}[H]
  \centering
  \includegraphics*[width=\textwidth]{./use_case_diagram.png}
  \caption{Use Case Diagram}
  \label{SopeOfProduct:ContextDiagram}
\end{figure}
\section{Use Case Table}
\sffamily
\begin{footnotesize}
  \renewcommand{\arraystretch}{1.4}
  \begin{longtable}[i i i L]{ p{.1\textwidth} p{.2\textwidth} p{.16\textwidth} p{.44\textwidth} }
    \caption                       % Caption für das Tabellenverzeichnis
        {Use Case Table} % Caption für die Tabelle selbs
        \\
    \toprule
    \textbf{UC ID} & \textbf{UC Name} & \textbf{Actors}  & \textbf{Description}\\
    \midrule
    \hypertarget{Ref:UC1}{UC\_1} & View vehicles & Decision Maker oder Sales Employee & Der Decision Maker/Sales Employee sieht in einer Liste alle Fahrzeuge für sein Projekt. \\
    \hypertarget{Ref:UC2}{UC\_2} & View vehicle information & Decision Maker, Sales Employee & Der Decision Maker oder Sales Employee wählt ein Fahrzeug aus und sieht sich die vorhandenen Information an. \\
    \hypertarget{Ref:UC3}{UC\_3}  & View fleet information & Decision Maker, Sales Employee & Der Decision Maker/Sales Employee sieht Statistiken über alle Fahrzeuge in seiner Flotte. \\
    \hypertarget{Ref:UC4}{UC\_4}  & View projects & Sales Employee & Der Sales Employee sieht die Projekte, welche er bearbeiten will. \\
    \hypertarget{Ref:UC5}{UC\_5} & Create project & Sales Employee & Der Sales Employee erstellt ein Projekt für einen neuen Entscheidungsträger. \\
    \hypertarget{Ref:UC6}{UC\_6} & Edit project & Sales Employee & Der Sales Employee bearbeitet die Projektinformationen. \\
    \hypertarget{Ref:UC7}{UC\_7} & Delete project & Sales Employee & Der Sales Employee löscht ein Projekt. \\
    \hypertarget{Ref:UC8}{UC\_8} & Reset project & Sales Employee & Der Sales Employee setzt ein Projekt auf eine Vorlage zurück. \\
    \hypertarget{Ref:UC9}{UC\_9} & Create template & Sales Employee & Der Sales Employee erstellt eine Vorlage aus einem Projekt. \\
    \hypertarget{Ref:UC10}{UC\_10} & Edit table & Sales Employee & Der Sales Employee bearbeitet die Liste der Fahrzeuge. \\
    \hypertarget{Ref:UC11}{UC\_11} & Create widget & Sales Employee & Der Sales Employee fügt die Visualisierung eines Datenpunkts für ein Fahrzeug oder eine Flotte hinzu. \\
    \hypertarget{Ref:UC12}{UC\_12} & Delete widget & Sales Employee & Der Sales Employee löscht die Visualisierung eines Datenpunkts für ein Fahrzeug oder eine Flotte. \\
    \hypertarget{Ref:UC13}{UC\_13} & Move widget & Sales Employee & Der Sales Employee passt die Position der Visualisierung eines Datenpunkts für ein Fahrzeug oder eine Flotte auf der Ansicht an. \\
    \bottomrule
  \end{longtable}
\end{footnotesize}
\rmfamily

\section{Individual Product Use Cases}
Um den Ablauf der Use Cases besser zu verstehen, werden für alle Use Cases Abbildungen des Prototypen gezeigt, der im Laufe des Design Sprints erstellt wurde, um die Anforderungen zu erheben. Der Prototyp besteht aus zwei Perspektiven. Die erste Perspektive ist die Ansicht des Entscheidungsträgers, wenn er mit der Anwendung inteagiert. Die zweite Ansicht ist die es Sales Mitarbeiter und Konfigurators des Systems. Im Folgenden werden die einzelnen Ansichten des Prototypen aus der Perspektive des Entscheidungsträgers gezeigt.


\subsection{Perspektive des Entscheidungsträgers}
\begin{figure}[H]
  \centering
  \includegraphics*[width=\textwidth]{./prototyp/Default-View.png}
  \caption{Perspektive Entscheidungsträger: Startseite}
  \label{DecisionMaker:Homepage}
\end{figure}

Die Abbildung \ref{DecisionMaker:Homepage} zeigt die initiale Situation, aus welcher der Entscheidungsträger immer startet, nachdem er sich mit seinen Accountinformationen an der Applikation anmeldet. Auf dieser Seite wird der Use Case \hyperlink{Ref:UC2}{UC\_2} durchgeführt und die Möglichkeit der Ablauf für die Use Cases \hyperlink{Ref:UC1}{UC\_1} und \hyperlink{Ref:UC3}{UC\_3} ermöglicht. 


\subsubsection{Use Case 1: View vehicles}
Das Szenario beginnt auf der initialen Ansicht in Abbildung \ref{DecisionMaker:Homepage}. Der Entscheidungsträger oder Sales Mitarbeiter sieht auf dieser Seite eine Tabelle mit mehreren Spalten, die alle Fahrzeuge beinhaltet, welche mit seinem Caruso-Marketplace-Konto verbunden sind. Die Tabelle beinhaltet mehrere Spalten zur genaueren Identifaktion der Fahrzeugs, beispielsweise den Hersteller oder den Status. Die Fahrzeuge lassen sich mit einem Klick auf den Spalten in aufsteigender oder absteigender Reihenfolge sortieren, damit die Fahrzeuge schnell gruppiert werden können. Außerdem ist es möglich spezifische Fahrzeuge zu suchen oder anhand der Filteroptionen Fahrzeuge auszublenden, die nicht betrachtet werden sollen.

\subsubsection{Use Case 2: View vehicle information}
Das Szenario beginnt auf der initialen Ansicht in Abbildung \ref{DecisionMaker:Homepage}. Der Entscheidungsträger oder der Sales Mitarbeiter wählt ein Fahrzeug über die Tabelle der Fahrezeuge aus. Wahlweise geschieht dies, nachdem ein Fahrzeug gesucht oder gefiltert wurde. Daraufhin befindet er sich auf der Fahrezugansicht in Abbildung \ref{DecisionMaker:DetailsRube}.

\begin{figure}[ht]
  \centering
  \includegraphics*[width=\textwidth]{./prototyp/Details-Rübe.png}
  \caption{Perspektive Entscheidungsträger: Fahrzeugansicht}
  \label{DecisionMaker:DetailsRube}
\end{figure}

Auf dieser Seite befinden sich alle Fahrzeuginformationen, zu dem vorher ausgewählten Fahrzeug. Die Datenpunkte, die von Caruso für das Fahrezeug zur Verfügung gestellt werden, sind in Weißen Kästchen, sogenannten Widgets zusammengefasst und logisch gruppiert. Ein Beispiel für solch ein Widget ist die allgemeine Fahrzeuginformation mit der VIN und dem Hersteller des Fahrzeugs. Jedes Widget beinhaltet eine Information darüber, wann das letzte Mal eine Aktualisierung durchgeführt wurde, damit der Entscheidungsträger weiß, wie aktuell die Daten sind. Für den Fall, dass manche Datenpunkte vom Hersteller nicht an Caruso geliefert werden und somit das Widget keine Informationen anzeigen kann, wird eine entsprechende Information erscheinen, wie in Abbildung \ref{DecisionMaker:NotSupported}.


\begin{figure}[ht]
  \centering
  \includegraphics*[width=5cm]{./prototyp/Not-Supported.png}
  \caption{Perspektive Entscheidungsträger: Nicht unterstützter Datenpunkt}
  \label{DecisionMaker:NotSupported}
\end{figure}

\subsubsection{Use Case 3: View fleet information}
Das Szenario beginnt auf der initialen Ansicht in Abbildung \ref{DecisionMaker:Homepage}. Von dieser Seite klickt der Entscheidungsträger oder der Sales Mitarbeiter auf den Reiter Flottenstatistik. Daraufhin öffnet sich die Ansicht auf Abbildung \ref{DecisionMaker:Fahrzeugstatistik}.
\begin{figure}[H]
  \centering
  \includegraphics*[width=\textwidth]{./prototyp/Fahrzeugstatistik.png}
  \caption{Perspektive Entscheidungsträger: Fahrzeugstatistik}
  \label{DecisionMaker:Fahrzeugstatistik}
\end{figure}
Auf dieser Seite befinden sich die flottenspezifischen Informationen aller registrieten Fahrzeuge in Form von Widgets. Wie auch in der Listenansicht ist hier das Filtern und Suchen möglich.

Die Widgets beinhalten beispielsweise eine Karte mit der Position aller Fahrzeuge, eine Statistik über die Hersteller der Fahrzeuge oder eine Liste mit vergangenen Ereignissen der Fahrzeuge. 

Zusätzlich zum Aktualisierungszeitpunkt geben diese Widgets ebenfalls an, wie viele Fahrzeuge die notwendigen Datenpunkte für das Widget liefern. Dies erhöht das Verständnis bei der Interpretation der Daten, da die genaue Menge an verwendeten Fahrzeugen bekannt ist.

\newpage
\subsection{Perspektive Sales Mitarbeiter}

\begin{figure}[ht]
  \centering
  \includegraphics*[width=\textwidth]{./prototyp/Project-Selection.png}
  \caption{Perspektive Entscheidungsträger: Fahrzeugansicht}
  \label{Configurator:ProjectSelection}
\end{figure}
Die Abbildung \ref{Configurator:ProjectSelection} zeigt die initiale Situation, aus welcher der Sales Mitarbeiter immert startet, wenn sie sich erfolgreich am System angemeldet haben und einen Use Case durchführen will.

\subsubsection{Use Case 4: view projects}
Das Szenario beginnt auf Abbildung \ref{Configurator:ProjectSelection}. Der Sales Mitarbeiter sieht die vorhandenen Projekte in Form von weißen Kacheln. Diese beinhalten Informationen über den Namen der Firma, sowie Informationen über die letzte Einsicht oder die letzte Änderung. Der Sales Mitarbeiter hat die Möglichkeit mithilfe der Suche die angezeigten Projekte zu reduzieren. Somit sieht er die Projekte, die ihn für seien Konfiguration am meisten interessieren.


\subsubsection{Use Case 5: Create project}

Das Szenario beginnt auf Abbildung \ref{Configurator:ProjectSelection}. Der Sales Mitarbeiter klickt auf die Projektkachel mit der Aufschrift Projekt erstellen.Daraufhin öffnet sich ein Popup wie in Abbildung \ref{Configurator:CreateProjectPopup}. 

\begin{figure}[ht]
  \centering
  \includegraphics*[width=7cm]{./prototyp/Projekt-Erstellen-Popup.png}
  \caption{Perspektive Entscheidungsträger: Fahrzeugansicht}
  \label{Configurator:CreateProjectPopup}
\end{figure}

Hier trägt der Sales Mitarbeiter ein Dataplace-Konto ein, welches er mit dem Projekt verknüpfen will, sowie weitere Dataplace-Konten die in der Lage sein sollen, dieses Projekt einsehen zu können. Außerdem fügt er das Logo für des Unternehmens sowie deren beiden Hauptfarben zur Personalisierung hinzu. Anschließend klickt der Sales Mitarbeiter auf Speichern, worauf sich eine Liste mit Vorlagen wie in Abbildung \ref{Configurator:Template} öffnet, aus denen das Projekt intial erstellt wird.

\begin{figure}[ht]
  \centering
  \includegraphics*[width=\textwidth]{./prototyp/Project-Selection-1.png}
  \caption{Perspektive Entscheidungsträger: Fahrzeugansicht}
  \label{Configurator:Template}
\end{figure}
In der Liste kann zusätzlich ein Vorlage über die Suche gefunden werden. Die Vorlagen besitzen eine Vorschau, damit der Konfigurator vor der Auswahl sieht, ob er sich für diese Vorlage entscheiden will. Der Konfigurator klickt eine Vorlage an. Daraufhin ist das Projekt erstellt.

\subsubsection{Use Case 6: Edit project}

Das Szenario beginnt auf Abbildung \ref{Configurator:ProjectSelection}. Der Sales Mitarbeiter klickt auf die vertikalen Punkte in der rechten oberen Ecke einer Projektkachel. Daraufhin verändert sich die Projektkachel wie in Abbildung \ref{Configurator:ProjectTileEdit}.

\begin{figure}[ht]
  \centering
  \includegraphics*[width=5cm]{./prototyp/ProjectTileEdit.png}
  \caption{Perspektive Configurator: Projektkachel editieren}
  \label{Configurator:ProjectTileEdit}
\end{figure}

Der Sales Mitarbeiter wählt Bearbeiten aus. Daraufhin öffnet sich ein Popup ähnlich zu Abbildung \ref{Configurator:CreateProjectPopup}, in welchem die Projektinformationen bearbeitet werden können. Sobald der Sales Mitarbeiter fertig ist, klickt er auf Speichern. Damit ist das Projekt editiert.

\subsubsection{Use Case 7: Delete project}

Das Szenario beginnt auf Abbildung \ref{Configurator:ProjectSelection}. Der Sales Mitarbeiter klickt auf die vertikalen Punkte in der rechten oberen Ecke einer Projektkachel. Daraufhin verändert sich die Projektkachel wie in Abbildung \ref{Configurator:ProjectTileEdit}. Der Sales Mitarbeiter klickt auf Löschen. Daraufhin verschwindet das Projekt aus der Liste der Projekte und das Projekt ist gelöscht.
\subsubsection{Use Case 8: Reset project}

Das Szenario beginnt auf Abbildung \ref{Configurator:ProjectSelection}. Der Sales Mitarbeiter klickt auf die vertikalen Punkte in der rechten oberen Ecke einer Projektkachel. Daraufhin verändert sich die Projektkachel wie in Abbildung \ref{Configurator:ProjectTileEdit}. Der Sales Mitarbeiter klickt auf Zurücksetzen. Daraufhin öffnet sich eine Auswahl von Vorlagen wie in Abbildung \ref{Configurator:Template}. Der Sales Mitarbeiter wählt eine Vorlage aus, auf welche er die Ansicht zurücksetzen will. Damit ist das Projekt zurücksetzt.

\subsubsection{Use Case 9: Create template}

Das Szenario beginnt auf Abbildung \ref{Configurator:ProjectSelection}. Der Sales Mitarbeiter klickt auf die vertikalen Punkte in der rechten oberen Ecke einer Projektkachel. Daraufhin verändert sich die Projektkachel wie in Abbildung \ref{Configurator:ProjectTileEdit}. Der Sales Mitarbeiter klickt auf Als Vorlage speichern. Daraufhin erhält er die Möglichkeit einen Namen für die Vorlage zu vergeben. Anschließend wird die Vorlage gespeichert und ist in der Liste der Vorlagen vorhanden.

\subsubsection{Use Case 10: Edit table}

Das Szenario beginnt auf Abbildung \ref{Configurator:ProjectSelection}. Der Sales Mitarbeiter klickt auf ein Projekt, dass er bearbeiten will. Daraufhin öffnet sich eine Projektansicht wie in Abbildung \ref{Sales:VehicleList}. 

\begin{figure}[H]
  \centering
  \includegraphics*[width=\textwidth]{./prototyp/Default-View-Gwen.png}
  \caption{Perspektive Sales Mitarbeiter: Flottenübersicht}
  \label{Sales:VehicleList}
\end{figure}

Die Ansicht ähnelt der des Entscheidungsträgers auf Abbildung \ref{DecisionMaker:Homepage}, jedoch besitzt diese Ansicht die Möglichkeit zum Bearbeiten der Fahrzeugtabelle. Der Sales Mitarbeiter klickt auf Spalten Bearbeiten. Daraufhin öffnet sich ein Popup wie in Abbildung \ref{Sales:EditTable}.

\begin{figure}[ht]
  \centering
  \includegraphics*[width=\textwidth]{./prototyp/Kopfzeile-Bearbeiten.png}
  \caption{Perspektive Sales Mitarbeiter: Fahrzeugtabelle bearbeiten}
  \label{Sales:EditTable}
\end{figure}

In dieser Ansicht ist der Sales Mitarbeiter in der Lage, die Reihenfolge der Spalteneinträge zu bearbeiten, sowie Elemente hinzuzufügen oder zu entfernen. Der Sales Mitarbeiter wählt z. B. das Modell aus und fügt es zu den angezeigten Elementen hinzu. Daraufhin klickt der Sales Mitarbeiter auf speichern und die Tabelle beinhaltet die neu festgelegten Spalten als Einträge.

\subsubsection{Use Case 11: Create widget}

Das Szenario beginnt auf Abbildung \ref{Configurator:ProjectSelection}. Der Sales Mitarbeiter klickt auf das Projekt, dass er bearbeiten will. Daraufhin öffnet sich die Fahrzeugübersicht aus Abbildung \ref{Sales:VehicleList}. Der Sales Mitarbeiter klickt auf ein Fahrzeug, wenn er fahrzeugspezifische Widgets hinzufügen will, oder auf die Flottenstatistik, wenn der flottenspezifische Widgets hinzufügen will. Die Ansicht für die Fahrzeugansicht sieht wie in Abbildung \ref{Sales:VehicleInformation} aus. 

\begin{figure}[ht]
  \centering
  \includegraphics*[width=\textwidth]{./prototyp/Details-Gwen-Config.png}
  \caption{Perspektive Sales Mitarbeiter: Fahrzeugansicht}
  \label{Sales:VehicleInformation}
\end{figure}

Die Ansicht ähnelt der des Entscheidungsträgers auf Abbildung \ref{DecisionMaker:DetailsRube}, jedoch beinhaltet sie weitere Funktionalität zum Hinzufügen oder Löschen eines Widgets sowie die Möglichkeit zum Rückgängig machen oder wiederholen einer Aktion.

Der Sales Mitarbeiter klickt auf Hinzufügen. Daraufhin öffnet sich eine Auswahl von möglichen Widgets, die für die Ansicht hinzugefügt werden können. Die Abbildung \ref{Sales:AddWidget} zeigt diese Ansicht.

\begin{figure}[ht]
  \centering
  \includegraphics*[width=8cm]{./prototyp/WidgetAdd.png}
  \caption{Perspektive Sales Mitarbeiter: Fahrzeuginformation Hinzufügen}
  \label{Sales:AddWidget}
\end{figure}

Der Sales Mitarbeiter wählt das entsprechende Widget aus und sieht eine Vorschau für dieses. Daraufhin klickt er auf Speichern und das Widget wird auf die Fahrezugansicht hinzugefügt.

Da der Ablauf der gleiche für die Fahrzeugstatistik ist, wird dies nicht als separater Use Case betrachtet. Sollte der Sales Mitarbeiter flottenspezifische Widgets hinzufügen wollen, klickt er In Abbildung \ref{Sales:VehicleList} auf Flottenstatistik und landet auf der Ansicht in Abbildung \ref{Sales:Fleet}.

\begin{figure}[ht]
  \centering
  \includegraphics*[width=\textwidth]{./prototyp/Fahrzeugstatistik-Gwen.png}
  \caption{Perspektive Sales Mitarbeiter: Flottenstatistik}
  \label{Sales:Fleet}
\end{figure}

Auf dieser Ansicht befindet sich wie in Abbildung \ref{Sales:VehicleInformation} die Möglichkeit zum Hinzufügen neuer Widgets. Der Sales Mitarbeiter klickt auf Hinzufügen und es öffnet sich eine Liste mit Widgets wie in Abbildung \ref{Sales:FleetWidget}

\begin{figure}[ht]
  \centering
  \includegraphics*[width=8cm]{./prototyp/WidgetAddFleet.png}
  \caption{Perspektive Sales Mitarbeiter: Flotteninformation hinzufügen}
  \label{Sales:FleetWidget}
\end{figure}

Die Ansicht ist ähnlich zu der Farhzeugwidgetliste aus Abbildung \ref{Sales:AddWidget}. Es unterscheiden sich lediglich die zu hinzufügbaren Widgets, da diese sich auf Flotteninformation, also aggregierte Daten beziehen. Wenn der Sales Mitarbeiter ein Widget ausgewählt hat und auf Speichern klickt wird das Widget in der Flottenstatisitk hinzugefügt.


\subsubsection{Use Case 12: Delete widget}
Das Szenario beginnt in Abbildung \ref{Configurator:ProjectSelection}. Der Sales Mitarbeiter klickt das Projekt an, in welchem er ein Widget entfernen will. Daraufhin befindet sich auf Abbildung \ref{Sales:VehicleList}. Wie im vorherigen Use Case, kann hier ein Widget in der Flottenstatisitk oder Der Fahrzeugansicht entfernt werden. Auf der Abbildung \ref{Sales:VehicleInformation} und \ref{Sales:Fleet} besitzen alle Widgets ein Mülltonen-Symbol in der rechten oberen Ecke. Sobald der Sales Mitarbeiter auf dies klickt, wird das entsprechende Widget aus der Ansicht entfernt.

\subsubsection{Use Case 13: Move widget}
Das Szenario beginnt in Abbildung \ref{Configurator:ProjectSelection}. Der Sales Mitarbeiter klickt das Projekt an, in welchem er ein Widget bewegen will. Daraufhin befindet sich auf Abbildung \ref{Sales:VehicleList}. Wie im vorherigen Use Case, kann hier ein Widget in der Flottenstatisitk oder Der Fahrzeugansicht bewegt werden. Auf der Abbildung \ref{Sales:VehicleInformation} und \ref{Sales:Fleet} sind die Widgets vorhanden. Wenn der Sales Mitarbeiter ein Widget anklickt und gedrückt hält ist er in der Lage dessen Position in einem Raster zu verschieben. Beim Loslassen der Taste wird das Widget eingerastet und dessen Position ist verschoben.






