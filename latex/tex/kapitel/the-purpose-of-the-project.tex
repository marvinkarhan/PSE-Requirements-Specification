\chapter{The Purpose of the Project}
"How might we help \gls{nontechnical} \glspl{stakeholder} to understand the value of telematics data with visualizations / dashboards?". This question was posed to students of the Mannheim University of Applied Sciences by the company Caruso GmbH at the start of the winter semester 2022/23 as part of the course "Projektsemester Software Engineering (PSE)". Over the course of two semesters, an application will be developed which will convince Caruso's potential customers to buy their product with easily understandable data visualisation.

\section{The User Business or Background of the Project Effort}
The company Caruso has been producing a standardised interface for \gls{telemetry} of connected cars from various manufacturers since 2017. Connected cars are vehicles which can share physically measured data (\gls{telemetry}) such as temperature or tire pressure through an internet connection.

Caruso enters into contracts with car manufacturers (\glspl{oem}) such as Mercedes or BMW to buy their \gls{telemetry}. This proprietary vehicle information is collected and standardised into a single interface, the \gls{api}. For example, \glspl{data} in the \gls{api} include the position of a vehicle or its next service date. External customers can then subscribe to \glspl{data} or packages of their vehicles. How they are billed depends on the \gls{data}, though it is often monthly. Some may be billed per car per month, some per access per month and some incur a flat fee per month. Caruso sells these \glspl{data} and packages to customers like auto repair shops, insurance companies or roadside assistance companies. 
% Caruso Sales-Mitarbeiter Workflow??

Many of these customers are medium-sized companies which need to be convinced of the value of this interface and the data before they are willing to invest in access to the \gls{dataplace}. These customers are called decision makers. They are the representatives of a company which decide to buy the data. Usually they are executives without a technical background, meaning that they don't understand the value of raw \glspl{data} alone. These decision makers are much more interested in their data being visualised in an easily understandable way that also shows its value. Especially important is live data of a vehicle such as its position changing as it is being driven.

Caruso currently does not offer an application with which live data can be explored by a \gls{nontechnical} user. Currently, scripts are used to gather data of a vehicle through the interface and accumulate it in an Excel or HTML report. These reports are manually created by multiple Caruso employees and have to be individually created for every customer.

\section{Goals of the Project}
The overarching goal of this project is as follows: "Through an improved workflow, \glspl{data} that are relevant to the decision maker can be visualised more quickly and comprehensively by the sales employee". The product being developed will aid the sales employee of Caruso to more quickly and efficiently develop presentations for potential customers and decision makers. This goal can be split into separate sub-goals: 
\begin{enumerate}
  \item The product visualises live information of the vehicles. This increases the decision maker's trust in Caruso.
  \item The product shows the \glspl{data} in a way that is comprehensible to \gls{nontechnical} decision makers. This makes the value of the data more readily apparent.
  \item The product is reusable for multiple decision makers. This reduces the time investment and manual work necessary for each data \gls{appetiser}.
  \item The product is easily configurable without technical knowledge. This will allow the \gls{nontechnical} employees of Caruso, such as a sales employee, to prepare an \gls{appetiser} without help of a developer.
\end{enumerate}

\section{Overall Solution}
Broadly speaking, the solution to the question posed by Caruso is a web-based application which hosts multiple dashboards. Each dashboard is configurable to fit the needs of individual decision makers. They include \glspl{data} from the decision maker's vehicles, visualised into an easily understandable form, for both individual vehicles and groups of vehicles (fleets). The solution will be called Carvis in the following chapters.