\chapter{The Purpose of the Project}
\begin{itemize}
  \item Projektsemesters über Zwei Semester
  \item Lösung für Caruso soll entwickelt werden
  \item Sogenannter "Data apettizer" soll es ermöglichen Kunden mehr
  \item Das Team eight eyes
\end{itemize}
Im Rahmen des Projektsemesters für Software Engineering im Wintersemester 2022/2023 und dem Sommersemester 2023 der Hochschule Mannheim soll für die Firma Caruso GmbH eine Anwendung zur

\section{The User Business or Background of the Project Effort}
\begin{itemize}
  \item Caruo ist ein unternehmen spezialisiert auf die standartisierung von Autodaten für "connected cars"
  \item Ein "conntected car" ermöglicht die Übertragung von Fahrzeuginformationen über das internet
  \item verschiedene Hersteller (Mercedes, BMW, etc.) bieten daten an und caruso nimmt diese und stellt eine api zur verfügung
  \item Caruso schließt also verträge mit den sogenannten OEMS (Herstellern) ab und bietet die schnittstelle
  \item Die Kunden von Caruso sind Unternehmen unterschiedlicher Art von Werkstätten, Versicherungen, Road Side Assistance
  \item Diese Kaufen die schnittstelle ein, um die Anwendeungen zu entwickeln, die ihnen widerrum helfen, ihre Kunden besser zu bedienen (Vllt. Beispiel mit Werkstatt)
  \item Carsuo muss ihre Kunden davon überzeugen, dass die Schnittstelle mit ihren Daten funktionieren und Möglichkeiten für Ideen bieten und dies möglich auch mit live informationen darstellen
  \item Aus Aussagen von Caruso geht hervor, dass viele Kunden den Mehrwert ihrer sogenannten Datenpunkte nicht wahrnehmen und keine verträge mit Caruso eingehen
  \item Der Prozess um Beispieldaten zur Überzeugung zu liefern bietet dabei weitere Probleme.
\end{itemize}
\section{Goals of the Project}
\begin{itemize}
  \item Where does our project help?
  \item what do we want to achieve
  \item long term goal?
\end{itemize}

\section{Overall Solution}
\begin{itemize}
  \item Die Anwendung ist ein Ansammlung webbasierter Dashboards, , welche die Datenpunkte einzelner und mehrerer Autos für einen Entscheidungsträger in einer verständlichen und leicht zugänglichen Weise darstellen. Mitarbeiter von Caruso sind in der Lage die Dashbaord für ihre jeweiligen Entscheidungsträger anzupassen, um so das Vertrauen zum Kunden besser aufbauen zu können.
\end{itemize}