\chapter{Usability and Humanity Requirements}
This chapter describes the requirements for how easy Carvis should be to use and how well it should meet the needs and expectations of its users.

\section{Ease of Use Requirements}

\snowcard{Usability and Humanity Requirements}
{Should}
{Fast creation of a \gls{appetiser}}
{For a sales employee, a \gls{appetiser} should be able to be created for the decision maker in under one hour}
{The tester creates a defined \gls{appetiser} from a real case in under one hour from scratch.}

\newline

\snowcard{Usability and Humanity Requirements2}
{should2}
{Fast creation of a \gls{appetiser}2}
{For a sales employee, a \gls{appetiser} should be able to be created for the decision maker in under one hour2}
{The tester creates a defined \gls{appetiser} from a real case in under one hour from scratch.2}

Ease of use is a prerequisite for the customers. All information should be quickly at a glance. 
Information should be limited to the most necessary for the home page and be quickly available at a glance.
It is not assumed that customers already have some experience with the digital world. 
Basic terms from the automotive industry such as \gls{vin}, on the other hand, should be known to all customers.

\section{Personalisation and Internationalisation Requirements}


As stated in \autoref{chap:lookandfeel}, Carvis should comply with the \gls{ci} of the customers. The logo and colours of Carvis should be adaptable to those of the customers to make them feel more at home.

For the sales employee, English is set as the default language. Since configurations are only created by Caruso where English is being spoken. I18n is therefore not necessary.

\section{Learning Requirements}
The decision maker will receive a short presentation on the application from the sales employee, but will also be able to use Carvis alone. 
Since the customers are to be convinced, a long learning phase should be refrained from.
The customer's view should be very understandable and easy to use. 

The sales employee is a Caruso employee. Since they will repeatedly use Carvis, a longer learning curve is to be expected. 