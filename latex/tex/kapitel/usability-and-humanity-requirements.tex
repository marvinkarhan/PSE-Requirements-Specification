\chapter{Usability and Humanity Requirements}

\section{Ease of Use Requirements}
For a configurator, the application should be able to be created for the decision maker in under 1 hour. 
Anything over that would not add value to the Caruso employee.


Ease of use is a prerequisite for the customer. All information should be quickly at a glance for the customer. 
Information should be limited to the most necessary for the overview and still be quickly available to the customer at a glance.
It is not assumed that customers already have some experience with the digital world. 
Basic terms from the automotive industry such as VIN, on the other hand, would be known to all customers.


\section{Personalization and Internationalization Requirements}
Like already declared in \autoref{chap:lookandfeel}, the application should be visually appealing by comply with the corporate branding of the customers.
Therefore, the colors should be customizable to the customer. Accordingly, even if there is a logo, it should be possible to replace it with that of the customer's company.

In the case of a configurator, English is set as the default language, since the configurator is only used by Caruso and English is a requirement there. I18n is therefore not necessary.

\section{Learning Requirements}
The decision maker will receive a short presentation on the application from Caruso, but will then use the application alone. 
Since the customers are to be convinced, a long learning phase should be refrained from.
The customer's application should therefore be very understandable and easy to use already at the beginning. 


In the case of a configurator, this is operated by Caruso himself. 
Since the configurator is used several times, a somewhat longer learning curve than with the customer may be possible. 