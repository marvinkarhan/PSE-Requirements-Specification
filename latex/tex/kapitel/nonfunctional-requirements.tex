\chapter{Non-Functional Requirements}
This chapter will introduce all non-functional requirements of the project. They are grouped according to the \href{https://www.volere.org/}{Volere template}\footnote[3]{https://www.volere.org/} and presented as snowcards, including an id, a name, a type, description, fit criterion and a priority. Like the functional requirements, they are prioritised according to the \gls{moscow}. 

\section{Look and Feel Requirements}
The following section will discuss the visual and tactile aspects of Carvis to ensure that it meets the desired standards for aesthetics and usability.
\label{sec:appearancerequirements}

\snowcard{Appearance Requirements}
{Should}
{Simple and understandable}
{Carvis should have a simple, easily understandable interface for \gls{nontechnical} users to not overwhelm them. It should feel \enquote{live}, using \glspl{data} that are as recent as possible.}
{A survey is launched which asks respondents to rate how easy the UI is to use based on looks alone.}

\snowcard{Style Requirements}
{Could}
{Quick and easy}
{The UI should be easy and quick to use, but still bring added value to the user.}
{A user can navigate to specific pages within 5 seconds.}

\section{Usability and Humanity Requirements}
This section describes the requirements for how easy Carvis should be to use and how well it should meet the needs and expectations of its users.

\snowcard{Usability and Humanity Requirements}
{Must}
{Fast creation of a \gls{appetiser}}
{For a sales employee, a \gls{appetiser} should be able to be created for the decision maker in under one hour.}
{The tester creates a defined \gls{appetiser} from a real case in under one hour from scratch.}

\snowcard{Usability and Humanity Requirements}
{Should}
{Quick recognisability}
{Ease of use is a prerequisite for the customers. Information should be limited to the most necessary for the home page and be quickly available at a glance. It is not assumed that customers already have some experience with the digital world. Basic terms from the automotive industry such as \gls{vin}, on the other hand, should be known to all customers.}
{The decision maker can use Carvis after watching the sales employee give their sales pitch.}

\snowcard{Personalisation and Internationalisation Requirements}
{Should}
{Language}
{For the sales employee, English is set as the default language. Since configurations are only created by Caruso where English is being spoken. I18n is therefore not necessary.}
{All texts are in English}

\snowcard{Learning Requirements}
{Should}
{Learning Phase}
{The decision maker will receive a short presentation using Carvis from the sales employee, but will also be able to use it alone. Customers will only spend a short time on Carvis, so learning should be kept to a minimum. The customer's view should be very understandable and easy to use. The sales employee will repeatedly use Carvis, thus a longer learning curve is to be expected.}
{A task catalog in the decision maker view should be created, which should be completed by several testers within a given time.}


\section{Performance Requirements}
This section covers the performance requirements of the product, including availability, capacity and scalability.

\snowcard{Scalability and Extensibility Requirement}
{Should}
{Adding New Content}
{As it is intended to service a wide range of customers with different interests, the ability to add new templates and widgets quickly is crucial.}
{Most new widgets and templates can be created in less than a work day and will appear in their respective lists automatically. Widgets that are more complicated, such as driver scoring, may take longer.}


\snowcard{Reliability and Availability Requirement}
{Must}
{Data Availability}
{Limitations in the availability of the \glspl{data} are completely on the side of the \gls{api} and \glspl{oem}. Only a steady connection to the \gls{api} must be insured.}
{An error message is displayed if the \gls{api} is unavailable.}

\snowcard{Reliability and Availability Requirement}
{Should}
{Uptime}
{Decision makers are unlikely to spend much time on Carvis by themselves, and likely only in short windows. This makes it all the more important that when they decide to use it that it is running and available.}
{Carvis should be running and available 99\% of the time.}


\section{Security Requirements}

This section covers the access and privacy requirements for the product.

\snowcard{Access Requirement}
{Will not}
{Multiple Viewers}
{Decision makers should have access to their cars that are connected via the \gls{dataplace} and should also be able to delegate access to other employees within their company. The decision maker should be able to decide who has access sensitive information and can perform certain actions. This functionality is already given within the \gls{dataplace}.}
{}

\snowcard{Privacy Requirement}
{Must}
{Configurator Vehicles}
{It is important to keep in mind that configurators do not have access to the cars registered to a decision maker. They can only access the cars that they are specifically assigned to. This should be considered when designing the configurator's view, as it may be necessary to include some example cars that can be used for testing purposes.}
{The configurator uses VirtualOEM from Caruso, which provides a set of sample cars that can be used for testing and development. This will help to ensure that configurators are able to effectively test and debug their configurations without needing access to actual cars.}

\snowcard{Privacy Requirement}
{Will Not}
{Consent Management}
{One important aspect of the privacy requirements in an application concerning connected cars is consent management. Consent management refers to the processes and systems in place for obtaining and documenting an individual's consent for the collection, use, and sharing of their personal data. In the case of Carvis, consent management is handled by Caruso and is therefore not subject of this project.}


