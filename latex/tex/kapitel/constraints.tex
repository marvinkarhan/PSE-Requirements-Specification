\chapter{Constraints}

In this chapter, we will explore the constraints that must be considered in the design of our product. These constraints represent limitations that apply to the entire product, and must be taken into account no matter how we choose to solve the problem at hand. In this section, we will discuss the description, rationale, and fit criterion for each constraint, and how they are written in the same format as functional and non-functional requirements. Understanding and adhering to these constraints will be crucial to the success of this project.

\section{Solution Constraints}

% Webapp
% \snowcard % Snowcard einbinden (Anpassungen in titelblatt.tex)
%    {TODO: Numver of the Requirements} % Nummer des Requirements
%    {F} % Art
%    {Hoch} % Priorität
%    {User Authentifizierung} % Titel
%    {Interview mit Abteilungsleiter} % Herkunft (Optional)
%    {F12} % Konflikte (Optional)
%    {Der Benutzer ist in der Lage sich über seinen
%     Benutzernamen und sein Passwort am System anzumelden} % Beschreibung
%    {Ein Benutzer kann sich mit seinem firmenweiten Benutzernamen und
%    Passwort über die Anmeldemaske anmelden und hat Zugriff auf die
%    Funktionen des Systems} % Fit-Kriterium (Optional)
%    {Benutzerhandbuch des Altsystems} % Material (Optional)

% umbenennen?
\section{Partner or Collaborative Applications}

As Carvis is a data appetizer for connected car data provided by Caruso. This constraint affects the data sources and the type of data that can be used. As the data provided by Caruso is ever evolving it is best to check the available data points on the Caruso dataplace platform (TODO: link to Caruso dataplace).

% Caruso Dataplace

% only use data for cars supported by caruso
% only use data provided by caruso

\section{Schedule Constraints}

% wir haben uns auf einen usecase (flottenmanagement) konzentriert
Time is a constraint imposed by the nature of this project as a student project. This constraint largely affects the content of the application, as it limits the resources we have in terms of the widgets we can include. In order to manage this constraint, we have focused on the fleet management use case, which includes widgets that are relevant for all use cases.

By focusing on the fleet management use case, we have been able to maximize the resources we have within the given time frame and create a product that meets the needs of our stakeholders to the best of our ability. This enables to deliver a product that can be used in a production environment.

% TODO: maybe include the image of caruso showing the most important usecases to highlight why whe chose fleet management


\section{Budget Constraints}
% gibt keine Budget Constraints (Bezug auf 3b. nehmen)
% Description: The product shall not be subject to budget constraints.
% Rationale: This is a demo application that will only be used for a short time frame, and as such, budget constraints do not need to be considered.
% Fit criterion: The product shall not exceed the allocated budget for the demo application.

One constraint that we will not need to consider in the design of our product is budget constraints. Because this is a demo application that will only be used for a short time frame as per User. As such, we do not need to worry about staying within a specific budget. By not having to worry about budget constraints, we will have more flexibility in our design choices and can focus on creating a product that meets the needs and expectations of the stakeholders.