\chapter{Constraints}

In this chapter, we will explore the constraints that must be considered in the design of our product. These constraints represent limitations that apply to the entire product, and must be taken into account no matter how we choose to solve the problem at hand. In this section, we will discuss the description, rationale, and fit criterion for each constraint.

\section{Partner or Collaborative Applications Constraints}

This chapter explores applications that the product will collaborate with, namely the \gls{dataplace}. By examining these partner applications, we can identify potential integration issues and design constraints. The dataplace can be found \href{https://www.caruso-dataplace.com/developer-zone/}{\emph{here}}\footnote{https://www.caruso-dataplace.com/developer-zone/}.

% \gls{dataplace}
% only use data for cars supported by caruso
The \gls{dataplace} provides a data catalogue and a \gls{api} documentation. The data catalogue provides a list of all available \glspl{data} and their description. The \gls{api} documentation provides a list of all available \gls{api} endpoints and their description and examples. The data catalogue and the \gls{api} documentation is available on the \gls{dataplace} developer zone. The data provided by Caruso is only available for cars that are supported by Caruso. The list of supported cars is available on the \gls{dataplace}.

% Examples of these applications can be provided through descriptions, models, or references to specifications that include all interfaces impacting the product
As Carvis is a data appetiser for connected car data provided by Caruso, this constraint affects the data sources and the type of data that can be used. As the data provided by Caruso is ever evolving it is best to check the available \glspl{data} on the \gls{dataplace} platform. 

\section{Schedule Constraints}

% wir haben uns auf einen usecase (flottenmanagement) konzentriert
Time is a constraint imposed by the nature of this project as a student project. This constraint largely affects the content of the application, as it limits the resources we have in terms of the \glspl{widget} we can include. In order to manage this constraint, we have focused on the fleet management \gls{usecase}, which includes \glspl{widget} that are relevant for all \glspl{usecase}.

By focusing on the fleet management \gls{usecase}, we have been able to maximize the resources we have within the given time frame and create a product that meets the needs of our \glspl{stakeholder} to the best of our ability. This enables to deliver a product that can be used in a production environment.

% TODO: maybe include the image of caruso showing the most important usecases to highlight why whe chose fleet management


\section{Budget Constraints}
% gibt keine Budget Constraints (Bezug auf 3b. nehmen)
% Description: The product shall not be subject to budget constraints.
% Rationale: This is a demo application that will only be used for a short time frame, and as such, budget constraints do not need to be considered.
% Fit criterion: The product shall not exceed the allocated budget for the demo application.

One constraint that we will not need to consider in the design of our product is budget constraints. Because this is a demo application that will only be used for a short time frame per user. As such, we do not need to worry about staying within a specific budget.