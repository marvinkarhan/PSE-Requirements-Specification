% -------------------------------------------------------
% Daten für die Arbeit
% Wenn hier alles korrekt eingetragen wurde, wird das Titelblatt
% automatisch generiert. D.h. die Datei titelblatt.tex muss nicht mehr
% angepasst werden.

% Titel der Arbeit auf Deutsch
\newcommand{\hsmatitelde}{Anforderungsdokument für ein Telemetriedatenvisualiserungsdashboard für die Caruso GmbH}

% Titel der Arbeit auf Englisch
\newcommand{\hsmatitelen}{Requirements Specification for a telemetry data visualization dashboard for Caruso GmbH}

% Weitere Informationen zur Arbeit
\newcommand{\hsmaort}{Mannheim}          % Ort
\newcommand{\hsmaautorvname}{Jan Diekhoff, Marvin Karhan, Jan Mayer, Joel Staubach}
\newcommand{\hsmaautorcontact}{eighteyes.info@gmail.com} % Kontaktadresse
\newcommand{\hsmaautornname}{} % Nachname(n)
\newcommand{\hsmadatum}{22.01.2023}      % Datum der Abgabe
\newcommand{\hsmajahr}{2023}             % Jahr der Abgabe
\newcommand{\hsmafirma}{Caruso GmbH} % Firma bei der die Arbeit durchgeführt wurde
\newcommand{\hsmabetreuer}{Prof. Dr. Peter Knauber, Hochschule Mannheim} % Betreuer an der Hochschule
\newcommand{\hsmazweitkorrektor}{Prof. Kirstin Kohler, Hochschule Mannheim}   % Betreuer im Unternehmen oder Zweitkorrektor
\newcommand{\hsmafakultaet}{I}    % I für Informatik oder E, S, B, D, M, N, W, V
\newcommand{\hsmastudiengang}{IB} % IB IMB UIB CSB IM MTB (weitere siehe titleblatt.tex)
\newcommand{\hsmaversion}{1.0} % Version

% Zustimmung zur Veröffentlichung
\setboolean{hsmapublizieren}{true}   % Einer Veröffentlichung wird zugestimmt
\setboolean{hsmasperrvermerk}{false} % Die Arbeit hat keinen Sperrvermerk

% "Creative Commons"-Lizenzen (https://creativecommons.org/)
% wenn Zustimmung zur Veröffentlichung und kein Sperrvermerk
%\renewcommand{\hsmacc}{by}
\renewcommand{\hsmacc}{by-sa}
%\renewcommand{\hsmacc}{by-nc-sa}

% -------------------------------------------------------
% Abstract
% Achtung: Wenn Sie im Abstrakt Anführungszeichen verwenden wollen, dann benutzen Sie
%          nicht "` und "', sondern \enquote{}. "` und "' werden nicht richtig
%          erkannt.

